\documentclass{article}

% these packages let you do math
\usepackage{amsmath}
\usepackage{amssymb}

% we need these packages for fancy R tables
\usepackage{booktabs}
\usepackage{float}
\usepackage{colortbl}
\usepackage{xcolor}

% these packages play with the spacing/margins of the document. Uncomment the commands on lines 16 and 17 to see what they do.
\usepackage{a4wide}
\usepackage{setspace}
\usepackage{geometry}
\usepackage{parskip}
%\doublespacing
%\geometry{margin=1.5in}

% this package helps us with including images. Setting the graphics path makes it easier to refer to things in the \includegraphics command.
\usepackage{graphicx}
\graphicspath{ {../figures/} }

% make some hyperlinks using the \href command
\usepackage{hyperref}
\hypersetup{
    colorlinks=true,
    linkcolor=black,
    urlcolor=blue
}

% set the author, title, and date of the document. \maketitle adds it to the document.
\author{Patrick Massey}
\title{Hidden Curriculum Assignment}
\date{02/18/2022}

\begin{document}
\maketitle

\section*{Data}
I use a subset of data from the 1997 National Longitudinal Survey of Youth, which contains longitudinal data on individuals from 1997 to 2019. In particular I use observations from 2002 that focus on the incarceration status of an individual for that year. When analyzing the data I remove any individuals for which there is no incarceration information available, or if they began the year already incarcerated. After cleaning the data I am left with 8,621 observations. I then create an dummy indicator value for if the individual is incarcerated at all during the year. This indicator variable \textit{incarcerated} will be the variable of interest for the model.
\newpage
\section*{Empirical Analysis} 
In this analysis I seek to estimate the probability of an individual becoming incarcerated based on race and sex. Looking at Figure \ref{fig:graph} below, we see that Black Males have the highest incarceration rate of almost 6\%. Males in general have a higher incarceration rate than Females. One anomaly is for the Mixed Race (Non-Hispanic) group. This should be seen as an small sample size issue because out of the 8,621 observations Mixed Race (Non-Hispanic) make up 81 observations.
\begin{figure}[H]
    \begin{center}
        \includegraphics[width=.85\textwidth]{incarceration_rate_by_racegender}
    \end{center}
    \caption{Incarceration Rate in 2002 by Race and Gender}
    \label{fig:graph}
\end{figure}
I summarize the graph in Table \ref{tab:tab:summarystats} below which shows the incarceration rate broken down by sex and race.
\input{../tables/incarceration_rates_by_racegender.tex}
The model we are seeking to estimate is shown below:
\begin{equation*}
Y_i = \beta X_i 
\end{equation*}
\newpage
Where $Y_i$ is a binary value for an individuals incarceration status and $X_i$ is a vector containing sex and race characteristics. I summarize the regression results in Table \ref{tab:regression} below.

\input{../tables/regress_incarceration_by_racegender.tex}

All variables are statistically significant at the 99\% level except for Mixed Race (Non-Hispanic) category which I think goes back to the small $n$ issue.
\end{document}
